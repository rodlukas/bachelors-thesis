Úspěšně jsem vytvořil webovou aplikaci na základě požadavků lektorky. Umožňuje evidovat klienty kurzů, jejich docházku, platby, historii a poskytuje přívětivé rozhraní pro úpravu těchto údajů. Součástí aplikace je také správa skupin a jejích kurzů, klientů a lekcí spolu s dalšími funkcionalitami usnadňujícími jednotlivé procesy.
V počátcích se objevilo nečekaně hodně problémů při zprovozňování a propojování zvolených technologiích. Některé plynuly z mé nedostatečné znalosti dotyčných frameworků, technologií a knihoven, jiné byly způsobeny špatnou dokumentací a část problémů souvisela s odlišnými přístupy jednotlivých frameworků. Všechny problémy se ale podařilo vyřešit a došlo tak zároveň k ověření toho, že příslušné technologie lze pro takovýto typ aplikace využít a těžit z jejich spojení.

Aplikace je nasazená a je lektorkou každodenně používána. Během akceptačního testování i ostrého provozu bylo vyladěno několik problémů a přidány některé další funkce pro pohodlnější správu klientů.

Samotný úspěch lze pozorovat jak z tvrzení lektorky, která si aplikaci chválí a je s ní nadmíru spokojena, tak i z ušetřeného času, který získala díky aplikaci připravené na míru jejím potřebám.

V plánu je rozšíření aplikace o další součásti s cílem vytvoření přívětivého, jednoduchého, ale mocného systému, který pokryje potřeby ve všech specifických procesech probíhajících v projektu.