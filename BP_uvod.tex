Úspěšný prvňáček\footnote{\url{https://uspesnyprvnacek.cz/}} (dále jen ÚP) je soubor kurzů vedených speciální pedagožkou PaedDr. Janou Rodovou. Cílem kurzů je pomoci budoucímu nebo nastupujícímu prvňáčkovi rozvíjet se tak, aby byl připraven na školní docházku. Kurzy se zabývají například prevencí selhávání v oblasti čtení a psaní, správným úchopem a držením tužky, rozvojem dovedností leváka nebo prací s neklidným či hyperaktivním dítětem. Některé z kurzů využívají kromě vlastních zkušeností lektorky také revoluční metodiky např. od ruského psychologa D. B. Elkonina nebo izraelského psychologa prof. Reuvena Feuersteina, PhD., tyto metodiky se do českých končin dostávají až v posledních měsících a letech a mezi rodiči jsou především díky jasným a viditelným úspěchům v rozvoji dovedností dítěte hravou formou velmi vyhledávány.

U projektu ÚP jsem již od úplného počátku (červenec 2014), kdy mimo jiné spatřily díky mně světlo světa jeho první webové stránky. Postupem času se projekt rozšiřoval až do fáze, kdy se na základě poptávky rozrostla nabídka kurzů tak, že bylo potřeba celý web od základů předělat. Tento fakt jsem využil i k rozšíření znalostí o novinky v HTML5, CSS3 a vybudoval jsem plně responzivní stránky přesně na míru projektu.

Ruku v ruce s tímto rozšířením samozřejmě opět vzrůstal počet klientů a bylo mimo jiné potřeba přehledně evidovat klienty a lekce. Jako nejrychlejší a v danou chvíli nejjednodušší řešení byla na počátku zvolena jednoduchá tabulka v Excelu doplněná o pár barev. Díky dalšímu nárůstu klientů a zvýšenému zájmu o skupinové kurzy je ale pro lektorku velmi složité udržet evidenci jakkoliv konzistentní, praktickou a přehlednou. Nemluvě o faktu, že ji téměř nelze rozšířit o další funkcionality a rozumnou evidenci skupinových lekcí. Práce s touto tabulkou je zbytečně zdlouhavá, neefektivní, data jsou duplikována ve více souborech a také ve více formách (papír) a jakákoliv změna vyžaduje pevné nervy.

Přáním lektorky, a tedy i mým cílem, je vytvořit webovou aplikaci, která umožní evidovat klienty, jejich docházku, skupiny, platby za lekce, historii lekcí klienta a další funkcionality zjištěné při analýze požadavků.

Mojí motivací pro vypracování práce na toto téma je především snaha využít technologie jako užitečný a podpůrný prvek projektu, díky kterému se jednotlivé každodenní procesy usnadní a ušetří lektorce čas. Druhou a neméně důležitou motivací je možnost prozkoumat, zmapovat a osvojit si některé z moderních technologií, ke kterým se v rámci této práce dostanu a rozšířit si tak své vědomosti a dovednosti.

V teoretické části nejprve zhodnotím aktuální stav evidence a pokusím se najít možná řešení a jejich výhody a nevýhody. Poté představím technologie, architektury a hostingy, které mohou být použity, zhodnotím je a zvolím ty vhodné pro tuto práci.

V praktické části zanalyzuji a popíši požadavky a související procesy v ÚP, navrhnu samotnou aplikaci a její části včetně struktury databáze a postupně uvedu své kroky při implementaci. Poté se budu věnovat průběhu testování včetně závěrečného akceptačního testování a provedeným úpravám. Na konec popíši nasazení výsledné aplikace na zvolený hosting a uvedu možná rozšíření v budoucnu.